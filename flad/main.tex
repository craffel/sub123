%%%%%%%% ICML 2022 EXAMPLE LATEX SUBMISSION FILE %%%%%%%%%%%%%%%%%

\documentclass[nohyperref]{article}

% Recommended, but optional, packages for figures and better typesetting:
\usepackage{microtype}
\usepackage{graphicx}
\usepackage{subfigure}
\usepackage{booktabs} % for professional tables

% hyperref makes hyperlinks in the resulting PDF.
% If your build breaks (sometimes temporarily if a hyperlink spans a page)
% please comment out the following usepackage line and replace
% \usepackage{icml2022} with \usepackage[nohyperref]{icml2022} above.
\usepackage{hyperref}


% Attempt to make hyperref and algorithmic work together better:
\newcommand{\theHalgorithm}{\arabic{algorithm}}

% Use the following line for the initial blind version submitted for review:
\usepackage{icml2023}

% If accepted, instead use the following line for the camera-ready submission:
% \usepackage[accepted]{icml2022}

% For theorems and such
\usepackage{amsmath}
\usepackage{amssymb}
\usepackage{mathtools}
\usepackage{amsthm}

% if you use cleveref..
\usepackage[capitalize,noabbrev]{cleveref}

%%%%%%%%%%%%%%%%%%%%%%%%%%%%%%%%
% THEOREMS
%%%%%%%%%%%%%%%%%%%%%%%%%%%%%%%%
\theoremstyle{plain}
\newtheorem{theorem}{Theorem}[section]
\newtheorem{proposition}[theorem]{Proposition}
\newtheorem{lemma}[theorem]{Lemma}
\newtheorem{corollary}[theorem]{Corollary}
\theoremstyle{definition}
\newtheorem{definition}[theorem]{Definition}
\newtheorem{assumption}[theorem]{Assumption}
\theoremstyle{remark}
\newtheorem{remark}[theorem]{Remark}

% Todonotes is useful during development; simply uncomment the next line
%    and comment out the line below the next line to turn off comments
%\usepackage[disable,textsize=tiny]{todonotes}
\usepackage[textsize=tiny]{todonotes}

% Alon - Custom section
% ===================================
\usepackage{array,multirow}
\usepackage{adjustbox}
\usepackage{enumitem}
\newcommand*{\argmax}{\operatornamewithlimits{argmax}\limits}
\newcommand{\alon}[1]{\textcolor{blue}{Alon: #1}}
\newcommand{\ww}[1]{\textcolor{red}{WW: #1}}
\newcommand{\ex}{EXP3}
\newcommand{\ucb}{UCB1}
\renewcommand{\paragraph}[1]{\noindent \textbf{#1}\quad~}

% ===================================



% The \icmltitle you define below is probably too long as a header.
% Therefore, a short form for the running title is supplied here:
\icmltitlerunning{Submission and Formatting Instructions for ICML 2022}

\begin{document}

\twocolumn[
\icmltitle{Improving Few-Shot Generalization by Exploring and Exploiting Auxiliary Data}

% It is OKAY to include author information, even for blind
% submissions: the style file will automatically remove it for you
% unless you've provided the [accepted] option to the icml2022
% package.

% List of affiliations: The first argument should be a (short)
% identifier you will use later to specify author affiliations
% Academic affiliations should list Department, University, City, Region, Country
% Industry affiliations should list Company, City, Region, Country

% You can specify symbols, otherwise they are numbered in order.
% Ideally, you should not use this facility. Affiliations will be numbered
% in order of appearance and this is the preferred way.
\icmlsetsymbol{equal}{*}

\begin{icmlauthorlist}
\icmlauthor{Firstname1 Lastname1}{equal,yyy}
\icmlauthor{Firstname2 Lastname2}{equal,yyy,comp}
\icmlauthor{Firstname3 Lastname3}{comp}
\icmlauthor{Firstname4 Lastname4}{sch}
\icmlauthor{Firstname5 Lastname5}{yyy}
\icmlauthor{Firstname6 Lastname6}{sch,yyy,comp}
\icmlauthor{Firstname7 Lastname7}{comp}
%\icmlauthor{}{sch}
\icmlauthor{Firstname8 Lastname8}{sch}
\icmlauthor{Firstname8 Lastname8}{yyy,comp}
%\icmlauthor{}{sch}
%\icmlauthor{}{sch}
\end{icmlauthorlist}

\icmlaffiliation{yyy}{Department of XXX, University of YYY, Location, Country}
\icmlaffiliation{comp}{Company Name, Location, Country}
\icmlaffiliation{sch}{School of ZZZ, Institute of WWW, Location, Country}

\icmlcorrespondingauthor{Firstname1 Lastname1}{first1.last1@xxx.edu}
\icmlcorrespondingauthor{Firstname2 Lastname2}{first2.last2@www.uk}

% You may provide any keywords that you
% find helpful for describing your paper; these are used to populate
% the "keywords" metadata in the PDF but will not be shown in the document
\icmlkeywords{Machine Learning, ICML}

\vskip 0.3in
]

% this must go after the closing bracket ] following \twocolumn[ ...

% This command actually creates the footnote in the first column
% listing the affiliations and the copyright notice.
% The command takes one argument, which is text to display at the start of the footnote.
% The \icmlEqualContribution command is standard text for equal contribution.
% Remove it (just {}) if you do not need this facility.

% \printAffiliationsAndNotice{}  % leave blank if no need to mention equal contribution
\printAffiliationsAndNotice{\icmlEqualContribution} % otherwise use the standard text.

\begin{abstract}
% What are we trying to do and why is it relevant?
Few-shot learning involves learning an effective model from only a few labeled datapoints.
Avoiding overfitting to the small training dataset leads to a difficult learning setting but also makes few-shot learning applicable to many important real-world settings.
% where collecting labeled data is difficult.
In this work, we focus on \textbf{F}ew-shot \textbf{L}earning with \textbf{A}uxiliary \textbf{D}ata (FLAD), a training paradigm that assumes access to auxiliary data during few-shot learning in hopes of improving generalizability.
% Why is this hard?
Introducing auxiliary data during few-shot learning leads to essential design choices
% , such as determining how to mix auxiliary and target data,
where hand-designed heuristics can lead to sub-optimal performance. 
% How do we solve it? What's our contribution
In this work, we focus on automated sampling strategies for FLAD and relate them to the explore-exploit dilemma that is central in multi-armed bandit settings.
Based on this connection we propose two algorithms, \ex{}-FLAD and \ucb{}-FLAD, and compare them with methods that either explore or exploit finding that the combination of exploration \textit{and} exploitation is crucial.
% How do we verify that our contributions have solved it?
Using our proposed algorithms to train T5 yields a 9\% absolute improvement over the explicitly multi-task pre-trained T0 model across 11 datasets.
\alon{Mention something about analysis here.}
\end{abstract}
\section{Introduction}

% Few-shot learning, challenges, and introduction to FLAD.
Few-shot learning is an attractive learning setting for many reasons: it promises efficiency in cost and time, and in some scenarios data is simply not available due to privacy concerns or the nature of the problem. However, few-shot learning is also a challenging setting that requires a delicate balance between learning the structure of the feature and label spaces while preventing overfitting to the limited training samples \citep{ravi2017optimization, 10.1145/3386252, https://doi.org/10.48550/arxiv.2203.04291}. One approach to improving the generalizability of models in the few-shot setting is \textbf{F}ew-shot \textbf{L}earning with \textbf{A}uxiliary \textbf{D}ata (FLAD), where auxiliary data is available that can be used to improve generalization on a target few-shot task \citep{10.1145/1015330.1015436, https://doi.org/10.48550/arxiv.1812.02224, esfandiarpoor2021, Verboven2022}.

 % \begin{figure}[t!]
 %     \centering
 %     \includegraphics[width=0.9\columnwidth]{images/bandit_visualization.pdf}
 %     \caption{A simple demonstration of few-shot learning with auxiliary data as multi-armed bandits. In step 1 a batch is sampled from auxiliary datasets, then in step 2 the model produces a reward. In step 3 the sampling distribution gets updated based on the reward.}
 %     \label{fig:simple_demo}
 % \end{figure}
 \begin{figure*}
     \centering
     \includegraphics[width=\textwidth]{images/flad_overview.pdf}
     \vspace{-5mm}
     \caption{Overview of our formulation of few-shot learning with auxiliary data as a multi-armed bandit problem, as described in Section \ref{sec:mab_to_flad}.}
     \label{fig:flad_overview}
 \end{figure*}

 
% What are the challenges of FLAD?
However, FLAD methods can introduce their own challenges, including increased algorithmic and computational complexity.
% Algorithmic complexity
Specifically, incorporating auxiliary data into the training algorithm introduces a large space of design choices for FLAD learning algorithms (e.g.\ how and when to train on auxiliary data).
Manually designing the curriculum for training on large quantities of auxiliary data is not feasible due to the combinatorially large search space, and hand-picking which auxiliary data to use (e.g.\ in-domain or on-task) may lead to sub-optimal results (see section 6.2 from~\citet{feta_albalak}).
% Computational complexity
Delegating such design choices to an algorithm can lead to better solutions, as demonstrated in the transfer learning \citep{vu-etal-2020-exploring, pruksachatkun-etal-2020-intermediate}, meta-learning \citep{10.5555/296635, bansal-etal-2020-self}, and multi-task literature \citep{Wu2020Understanding, aghajanyan-etal-2021-muppet}. However, such algorithmic design choices require additional computation, motivating the search for efficient methods as the quantity of auxiliary data increases.

% Our FLAD setting and desiderata for a FLAD algorithm
In this work, we consider the FLAD setting where auxiliary data is divided into separate datasets and training occurs simultaneously over both the target and auxiliary datasets. We desire a FLAD algorithm that

% \begin{enumerate}[topsep=0pt,partopsep=0pt,parsep=0pt]
\begin{enumerate}[nosep]
    \item makes no assumptions on available auxiliary data a-priori (in-domain, on-task, quality, quantity, etc.),
    \item continuously updates beliefs on importance of auxiliary data, and
    \item adds minimal memory and computational overhead.
\end{enumerate}

% How do we relate our problem to MAB?
In this work, we design algorithms that satisfy our desiderata by drawing inspiration from the central problem in multi-armed bandit (MAB) settings: the exploration-exploitation trade-off \citep{macready1998bandit, simpkins2008optimal}. We relate the set of auxiliary datasets to the arms of a MAB and tailor the classic \ex{} \citep{auer2002nonstochastic} and \ucb{} \citep{auer2002finite} algorithms to fit the FLAD framework by designing an efficient gradient-based reward signal. Figure \ref{fig:flad_overview} provides a basic illustration of how we formulate FLAD as an MAB problem.

% What does this paper do?
Our study presents two efficient algorithms, \ex{}-FLAD and \ucb{}-FLAD, which enhance the generalization capabilities of a machine learning model in few-shot settings. To empirically validate our approaches, we utilize readily available auxiliary datasets from P3 \citep{bach-etal-2022-promptsource} which may be in- or out-of-domain and related or unrelated to the target task. We evaluate our methods on the same held-out tasks as T0~\citep{sanh2022multitask} and show that, when using the same collection of auxiliary datasets, our algorithms outperform T0 (which simply concatenates the constituent datasets of P3) by 5.1\% (\ex{}-FLAD) and 5.6\% (\ucb{}-FLAD) absolute. Furthermore, incorporating all available datasets in P3 (i.e.\ not just those used to train T0) increases the improvement to 7.6\% and 9.1\% respectively.

% Summarization of contributions/findings

In summary, our main contributions are:
\begin{itemize}[nosep]
    \item We connect FLAD to the MAB setting and focus on the exploration-exploitation trade-off.
    \item We design two algorithms, \ex{}-FLAD and \ucb{}-FLAD that adapt existing MAB methods to the FLAD setting and propose a reward function that is simple and efficient (in both space and computational costs). 
    \item We present experimental results that validate that our methods improve few-shot performance of pretrained language models and show that strategies that employ only exploration or exploitation lead to sub-optimal performance.
    \item We perform two case studies to better understand why \ex{}-FLAD and \ucb{}-FLAD outperform baselines.
\end{itemize}

\section{Background}
In this section we first informally describe the few-shot learning with auxiliary data (FLAD) problem. Then, we present and define the goals of the multi-armed bandit setting. Next, we present the adversarial bandit setting, connect it to FLAD, and review the \ex{} algorithm \citep{auer2002nonstochastic} used to solve it. Finally, we present the \ucb{} algorithm \citep{auer2002finite}, used to solve a more constrained MAB setting.

% Intro to FLAD, which version of FLAD are we focused on
\subsection{Few-shot Learning with Auxiliary Data}
% FLAD is a broad category
Few-shot learning with auxiliary data (\textbf{FLAD}) aims to improve generalization when training on a small quantity of target data $\mathcal{D}_{\mathcal{T}}$ by making use of a much larger quantity of (possibly) related auxiliary data $\mathcal{D}_{\mathcal{A}}$. The auxiliary data may be labeled or unlabeled, be cleanly differentiated as separate datasets/tasks, or many other properties. The goal of FLAD is to optimize a model for the distribution underlying $\mathcal{D}_{\mathcal{T}}$.

Under this definition, a great deal of prior work can be seen to focus on FLAD.
Some past work sets up training in stages, such as STILTS~\citep{phang2018sentence}, meta-learning~\citep{bansal-etal-2020-self}, or multi-task learning~\citep{aghajanyan-etal-2021-muppet}, where the first stage generally uses all of the auxiliary data equally and the second stage includes only the target data. Other approaches have utilized retrieval-augmented models that match auxiliary data to large quantities of unlabeled target data such as in ReCross~\citep{Lin2022UnsupervisedCG} and DEFT~\citep{Ivison2022DEFT}. Alternatively, some works scale the loss of auxiliary datasets as in~\citet{Verboven2022}, or include unsupervised auxiliary data as in~\citet{deryAANG2022}.

To narrow down the set of methods that we consider for satisfying our desiderata, we focus on the setting where -- crucially -- the final stage of training includes simultaneous training on both supervised auxiliary and target data. Henceforth, we use FLAD to refer specifically to this setting.


% Brief intro to MAB
\subsection{Multi-Armed Bandits}

The Multi-Armed Bandit (\textbf{MAB}) setting is a problem from machine learning where a learner interacts with an environment over $N$ rounds by following a policy $\pi$. At each round $t$ the learner chooses one of the environment's $K$ arms, $a\in\mathcal{A}$ where $K=\vert\mathcal{A}\vert$, after which the environment provides a reward $R_{t}$. Rewards for unplayed arms are not observed. The goal of the learner is to adopt a policy $\pi$ that selects actions that lead to the largest cumulative reward over $N$ rounds, $R=\sum_{t=1}^{N}R_{t}$.
% To evaluate a policy, regret is calculated as the difference between the total expected reward of an optimal policy and the total expected reward collected by the learner \alon{either don't mention regret if not used elsewhere, or include more information on regret}.
In this work we assume a finite $K$ and that the underlying reward distribution of each arm may have a variety of properties (e.g.\ stochasticity or stationarity) depending on the exact scenario, leading to differing optimal policies~\citep{lattimore_szepesvári_2020}.

% Adversarial MAB and Exp3
\paragraph{Adversarial MAB}
% Adversarial MAB details
The adversarial MAB setting assumes that the reward-generating process is controlled by an adversary. This assumption allows for modelling non-stationary and highly stochastic reward signals.
% and makes almost no assumptions about the reward-generating process.
We will later show why our FLAD formulation fits into this setting.
% This is opposed to the previously discussed setting for the \ucb{} algorithm, which makes an assumption that rewards are gaussian random variables, and the mean reward of individual arms are (mostly) stationary.
Under this setting, it is assumed that an adversary is given access to the learner's policy $\pi$ and determines the sequence of rewards, $(R_{a,t})_{t=1}^{N}$, for each arm prior to play \citep{auer1995gambling}. At each turn $\pi$ determines a distribution over actions, $p(\mathcal{A})$, and an action is sampled from the distribution, $a\sim p(\mathcal{A})$.
See \citet{lattimore_szepesvári_2020} for further details.

% Background on EXP3
\paragraph{The \ex{} Algorithm}
The \ex{} algorithm (``Exponential-weight algorithm for Exploration and Exploitation'') targets the adversarial multi-armed bandit problem \citet{auer2002nonstochastic} by choosing arms according to a Gibbs distribution based on the empirically determined importance-weighted rewards of arms. To allow for exploration, \ex{} mixes the Gibbs distribution with a uniform distribution.

% Importance weighting and Gibbs distribution formulation
Formally, let the exploration rate be $\gamma \in (0,1]$. At round $t$, $\pi$ defines the probability of selecting a given arm, $a\in\mathcal{A}$, as a linear combination of Gibbs and uniform distributions
\begin{equation}
\label{eq:exp3_sampling}
    p_{t}(a) = (1-\gamma)\dfrac{\exp(\gamma\hat{R}_{a,t-1}/K)}{\sum_{a\prime}\exp(\gamma \hat{R}_{a\prime,t-1}/K)}+\frac{\gamma}{K}
\end{equation}
where the importance weighted reward $\hat{R}_{a,t}$ is calculated as 
\begin{equation}
\label{eq:importance_weighted_reward}
    \hat{R}_{a,t} = \hat{R}_{a,t-1} + \frac{R_{a,t}}{p_{t-1}(a)}
\end{equation}
and $R_{a,t}$ denotes the observed reward. All unplayed arms, $a\prime\neq a$ have unchanged importance weighted rewards; $\hat{R}_{a\prime,t}=\hat{R}_{a\prime,t-1}$.

% Algorithm
Algorithmically, \ex{} takes the following steps at each round. First, calculate the sampling distribution $p_{t}$ and sample an arm from the distribution. Then a reward $R_{a,t}$ is observed and the algorithm updates the importance weighted reward $\hat{R}_{a,t}$ for the played arm.

Informally, the use of an importance-weighted estimated reward compensates the rewards of actions that are less likely to be chosen, guaranteeing that the expected estimated reward is equal to the actual reward for each action. \ex{} is designed to be nearly optimal in the worst case, but due to the exploration rate it will select ``bad'' actions at a rate of $\gamma / K$. The exploration of \ex{} combined with importance-weighting allows the policy to handle non-stationary reward-generating processes.

% UCB1 algorithm, regret, optimality, etc.
\paragraph{The \ucb{} Algorithm}
% Background on UCB
While the adversarial setting makes almost no assumptions about the reward-generating process and therefore maintains its performance guarantees under almost any circumstances, it can be outperformed in settings that \textit{are} constrained. In this section we assume that the reward generating processes are stationary Gaussian distributions.
% While we will later see that this is not a guarantee in the FLAD setting, we will show that weakened versions of these assumptions hold.
A common policy used to solve this MAB problem is the Upper Confidence Bound (\ucb{}) algorithm, which assigns each arm a value called the upper confidence bound based on Hoeffding's inequality~\citep{auer2002finite}. The \ucb{} algorithm is based on the principle of \textit{optimism in the face of uncertainty}, meaning that with high probability the upper confidence bound assigned to each arm is an overestimate of the unknown mean reward.

% Upper confidence bound formulation
Formally, let the estimated mean reward of arm $a$ after being played $n_{a}$ times be $\hat{R}_{a}$ and the true mean reward be $R_{a}$, then
\[
\mathbb{P}\bigg(R_{a} \ge \hat{R}_{a} + \sqrt{\frac{2\ln (1/\delta)}{n_{a}}}\bigg) \le\delta \quad\forall \delta\in (0,1)
\]
derived from Hoeffding's inequality (following equation 7.1 of~\citet{lattimore_szepesvári_2020}), where $\delta$ is the confidence level that quantifies the degree of certainty in the arm. In this work we let $\delta = 1/t$ where $t$ is the number of rounds played, shrinking the confidence bound over rounds. Thus, we define the upper confidence bound for arm $a$ at turn $t$ as
\begin{equation}
\label{eq:ucb}
UCB_{a,t} = 
\begin{cases}
\infty,& \text{if } n_{a}=0 \\
\hat{R}_{a}+\sqrt{\frac{2\ln t}{n_{a}}},& \text{otherwise}
\end{cases}
\end{equation}

% Algorithm
Algorithmically, \ucb{} takes the following steps at each round. First, the \ucb{} policy plays the arm with largest upper confidence bound, $a^{*}=\arg\max_{a\in\mathcal{A}}UCB_{a,t}$. Next a reward, $R_{a^{*},t}$, is observed and the algorithm updates $\hat{R}_{a^{*}}$ (the estimated mean reward for $a^{*}$) and the upper confidence bounds for all $a$. Informally, this algorithm suggests that the learner should play arms more often if they either 1.\ have large expected reward, $\hat{R}$, or 2.\ $n_{a}$ is small because the arm is not well explored.

% Regret
% The \ucb{} algorithm has a worst-case regret of $O(\sqrt{K n \ln n})$.

% Weighting method
% Given a target dataset $\mathcal{D}_{\mathcal{T}}$, with limited samples, $K$ labeled auxiliary datasets, $\mathcal{D}_{\mathcal{T}}$, and a model $\mathit{f}_{\theta}$, the goal of FLAD is to train $\mathit{f}$ to achieve high performance on the target dataset by simultaneously minimizing the losses
% \[
% \mathcal{L}_{\mathcal{A}\cup\mathcal{T}} = w_{\mathcal{T}}\mathcal{L}_{\mathcal{T}} + \sum_{a\in\mathcal{A}}w_{a}\mathcal{L}_{a}
% \]
% where the weights, $w_{\mathcal{T}},w_{\mathcal{a}}$, are determined by a policy. The policy can be fixed prior to training, adjusted on a predetermined schedule, or learned in an online fashion. In this definition we make no assumptions about the quality of auxiliary data with regards to transfer to the target dataset.
% \alon{Maybe this entire subsection isn't required? State FLAD in it's simplest terms and then move on!}
% Describe multi-task learning and meta-learning under this framework?
\section{From MAB to FLAD}
\label{sec:mab_to_flad}

% (1) How can FLAD be formulated as an MAB problem
In this work we formulate few-shot learning with auxiliary data (FLAD) as a multi-armed bandit (MAB) problem as follows. Assume auxiliary data $\mathcal{D}_{\mathcal{A}}$ is split into individual datasets $\mathcal{D}_{a}$ and at each round of training a learner updates it's policy $\pi$ over the auxiliary datasets. Assume that at each round a batch of data is sampled from a single dataset $\mathcal{D}_{a}$ selected by the policy, i.e.\ $\{\mathbf{x},\mathbf{y}\} \sim \mathcal{D}_{a}$. Then, the model being trained $f_\theta$ is updated through a gradient w.r.t. $\theta$ calculated over the sampled batch using a task-appropriate loss function as $\nabla_{a}=\nabla_{\theta}\mathcal{L}(\mathit{f}_{\theta},\mathbf{x},\mathbf{y})$. For practical purposes of optimizing large neural networks with stochastic gradient descent-based methods, we let $G$ be the number of rounds between model updates to allow for varying batch sizes and for multiple auxiliary datasets to occur in a single model update. For simplicity, after every $G$ rounds the model also computes the target gradient, $\nabla_{\mathcal{T}}=\nabla_{\theta}\mathcal{L}(\mathit{f}_{\theta},\mathcal{D}_{\mathcal{T}})$, and updates model parameters w.r.t. $\nabla_{\mathcal{T}}+\sum_{a\in\mathcal{A}}\nabla_{a}$.

\paragraph{Designing the Reward Function}
% Considerations
% A central feature of the MAB setting is that the environment provides a reward after each decision made by the learner. There are multiple aspects to consider when designing a reward for FLAD, and we design our reward function with the following considerations: 1. it should utilize information intrinsic to the model and the losses being optimized, not a metric external to the model (e.g.\ accuracy or BLEU), 2. it should incur minimal space and computational costs above standard training procedures, and 3. it should be proven as a good approximation to transfer.
We design the reward function with our desiderata in mind. To ensure that our algorithm adds minimal memory and computational overhead we consider rewards that utilize information intrinsic to the model and the losses being optimized, not an external metric (e.g.\ accuracy or BLEU).

% Our reward
In this work we adopt a gradient-based reward inspired by previous works~\citep{https://doi.org/10.48550/arxiv.1812.02224, NEURIPS2019_0e900ad8, yu2020gradient, wang2021gradient}. Formally, at turn $t$ let $\mathcal{D}_a$ be the auxiliary dataset selected by the learner, then we define our reward function as
\begin{equation}
\label{eq:reward}
R_{a,t} = \frac{\nabla_{a}\cdot\nabla_{\mathcal{T}}}{\|\nabla_{a}\|\|\nabla_{\mathcal{T}}\|}
\end{equation}
i.e.\ the cosine similarity between the gradients of the sampled auxiliary dataset batch and the whole target dataset. This reward adds minimal computational complexity at each round (three vector dot products, two square roots, and two scalar multiplications). Additionally, while this reward adds space requirements,
% storage of the target gradient and $G$ auxiliary gradients at a time,
we show later that this can be reduced to at most $(G+1)*2.3$\% of the full model parameters.
% Furthermore, this metric has been thoroughly proven in previous works~\citep{https://doi.org/10.48550/arxiv.1812.02224, NEURIPS2019_0e900ad8, yu2020gradient, wang2021gradient}.


% Now, to allow the learner from MAB to make such decisions, we need to decide on a reward. We use cosine similarity between gradient of auxiliary dataset and target dataset, inspired by previous methods of quantifying task relatedness \citep{https://doi.org/10.48550/arxiv.1812.02224, vu-etal-2020-exploring}. \citet{pruksachatkun-etal-2020-intermediate} utilize probing tasks to determine transferability across tasks, but this is not computationally feasible for an online algorithm.

% (2) Assumptions (differences between standard MAB and our setting)
\subsection{Modelling Assumptions, Implications, and Consequences}
\label{sec:modelling_assumptions}
% Assumptions:
% Implicit in our formulation of FLAD are a few assumptions. \alon{Shorten this section or move to discussion section.}

% Adversarial setting
% \paragraph{FLAD as an Adversary} Firstly, the formulation we follow necessitates modelling the environment as an adversary. This is due to three factors: the non-convex loss landscape of deep neural networks, the use of stochastic gradient descent-based methods for optimization, and our use of gradient alignment as a reward function. These factors imply that we cannot guarantee that the rewards for each arm are stationary or gaussian. However, we will later empirically show that gradient alignments maintain a weak form of stationarity.
\paragraph{FLAD as an Adversary} First, the formulation we follow necessitates modelling the environment as an adversary. This is due to three factors: the highly non-convex loss landscape of deep neural networks, the use of stochastic gradient descent-based optimization, and our use of gradient alignment as a reward function. These factors imply that the rewards for each arm cannot be guaranteed to be stationary, independent, or gaussian. However, we will later empirically show that gradient alignments maintain a weak form of stationarity.

% Multi-task model
% \paragraph{Parameter Sharing}
% In addition, our reward function requires that the model $\mathit{f}$ share parameters across datasets. We believe this is not overly restrictive thanks to the ongoing trend of unifying input formats so that a single model can handle many tasks. For example, in NLP, works have focused on a unified text-to-text format~\citep{McCann2018decaNLP,radford2019language,raffel2020t5, NEURIPS2020_1457c0d6}, and other works have extended this idea to multi-modal settings where e.g.\ text and images can be fed into a single model \citep{wang2022ofa, alayrac2022flamingo}.
% Such models are good candidates for our reward formulation.
\paragraph{Parameter Sharing}Next, our reward function requires that the model $\mathit{f}$ shares parameters across datasets. We believe this is not overly restrictive due to the trend of unifying input formats for single models to handle many tasks (e.g.\ text-to-text~\citep{McCann2018decaNLP,radford2019language,raffel2020t5, NEURIPS2020_1457c0d6} and multi-modal settings~\citep{wang2022ofa, alayrac2022flamingo}).

% Efficiency-vs-accuracy tradeoff
% \paragraph{Mini-Batch SGD}
% Next, we assume that the gradient from a single mini-batch from an auxiliary dataset is a reasonable approximation of the gradient w.r.t. the full dataset. This assumption allows us to run the algorithm efficiently, rather than requiring a pass over full auxiliary datasets. We believe this to be a reasonable assumption as over the course of many sampling steps approximation errors tend to balance each other out and this has proven to improve training efficiency as has been well demonstrated with mini-batch SGD.
% Similarly, we have a tradeoff between efficiency and accuracy in our reward function. While it is possible to use validation metrics as a reward, this would require validating after every mini-batch, increasing the time-complexity of the training. Additionally, this would require us to use only a single mini-batch per gradient update, as opposed to our currently flexible setup which updates every $G$ steps.
\paragraph{Mini-Batch Reward Calculation}Additionally, to improve efficiency, we assume that the gradient of a single mini-batch from an auxiliary dataset is a reasonable approximation of the gradient w.r.t the full dataset. We believe this to be a reasonable assumption as over the course of many sampling steps approximation errors tend to balance each other out and this has proven to improve training efficiency as has been well demonstrated with mini-batch SGD.

% \paragraph{Discrepancy Between MAB Metric and Test Metric}
% Finally, in traditional bandit settings the success of an algorithm is directly measured by the reward being optimized. However, in our setting the reward we receive from playing an arm (e.g.\ gradient similarity) may not the same metric that we aim to optimize (e.g.\ cross-entropy, accuracy, or ROUGE-L on a held-out validation set of the target task). Furthermore, although we measure algorithmic success on a validation set, we actually desire for the model to generalize to an unseen test set, further removing our reward from the underlying metric of interest.
\paragraph{Discrepancy Between Reward and Evaluation Metric}Finally, the success of an algorithm in traditional bandit settings is directly measured by the reward being optimized. However, in our setting the reward we receive from playing an arm may not be the same metric that we aim to optimize (e.g.\ cross-entropy, accuracy, or ROUGE-L).

% UCB1 assumes rewards are normally distributed

% MAB problems are designed based on the fact that there will always be a single arm with highest average reward over any time horizon. While that will be true for receiving rewards, in FLAD, maximizing similarity is a proxy for our true goal, improving a metric on a held out validation set. In fact, in an extreme case where the learner finds some data which is exactly the same as our target dataset, then we end up with the same performance as if we fine-tuned directly on the target data only. In fact, this leads to worse performance because we are in a few-shot setting, and we want to introduce useful noise through our auxiliary tasks.

% \citet{Kim2020TheLC} show that the self-attention mechanism is not Lipschitz for an unbounded input domain. In this and many recent works in NLP, models based on the transformer architecture are used which utilize self-attention. Although we can bound the change in model parameters during the learning process, because of the non-Lipschitz property of self-attention, we cannot make any guarantees on how model predictions will change over the course of training. This leads us to believe that our FLAD setting is most similar to the adversarial MAB formulation.

% \paragraph{Challenges for the Learner}
% % Challenges:
% Under our formulation, the learner faces the following challenges:
% \begin{enumerate}
%     \item Efficiently determining the effect of each auxiliary data batch on the target task
%     \item Balancing selection from the most rewarding datasets while taking into consideration the dynamics of training (exploration-exploitation)
% \end{enumerate}

% As we have demonstrated, our formulation for FLAD fits very well into the MAB framework, and so in the next section we will introduce our variations on the \ex{} and \ucb{} algorithms to solve the FLAD problem.
\section{Adapting MAB algorithms for FLAD}
In this section we describe our variations on the \ex{} and \ucb{} algorithms based on how these algorithms do and do not fit the FLAD setting. We present pseudo-code for our methods in Algorithms \ref{alg:exp3} and \ref{alg:ucb1}.

% \alon{$t$ has been omitted in algorithm pseudo-code where obvious from context}
\subsection{\ex{} for FLAD}

% Adjustments to \ex{}
Recall that the \ex{} algorithm is designed for the adversarial MAB setting, where no assumptions are made on the reward generation process and in Section~\ref{sec:modelling_assumptions} we argue that the adversarial MAB setting is an appropriate modelling choice for FLAD.
However, in FLAD we can make some assumptions to weaken the power of the adversary and improve the quality of our learner.

% Very bad datasets will likely stay very bad, exploration rate decay
First, we assume that the helpfulness or harmfulness of training on a particular auxiliary dataset (in terms of the impact on target dataset performance) will be relatively consistent throughout training.
% This is a reasonable assumption because if our target gradient is orthogonal or opposite direction of an auxiliary gradient prior to being trained at all, it's likely that their solutions are quite far from each other in parameter space.
Of course, because we randomly sample batches during training, the reward generating process is noisy and we cannot know for sure early on whether a dataset is actually helpful or harmful.
To model this assumption, we use a decaying exploration rate, as proposed by \citet{pmlr-v24-seldin12a}.
% Instead of using a constant exploration rate, $\frac{1}{K}$, we use an exploration rate that changes over time.
Recall that $K$ is the number of auxiliary datasets. Then at the beginning of round $t$ we use an exploration rate of
\begin{equation}
\label{eq:exploration_rate}
    \mathcal{E}_{t} = \min \Bigl\{\frac{1}{K}, \sqrt{\frac{\ln K}{K\cdot t}} \Bigr\}
\end{equation}
Further, we adjust the sampling distribution from equation \ref{eq:exp3_sampling} to include the new exploration rate as
\begin{equation}
\label{eq:our_exp3_sampling}
    p(a) = (1-K\mathcal{E}_{t})\frac{\exp(\mathcal{E}_{t-1}\hat{R}_{a})}{\sum_{a^{\prime}} \exp(\mathcal{E}_{t-1}R_{a^{\prime}})}+\mathcal{E}_{t}
\end{equation}
Under our assumptions, this decaying rate now allows for an initial exploratory period after which the learner will select ``bad`` actions at a rate of $\mathcal{E}_{t}=\sqrt{\frac{\ln K}{K\cdot t}}$.

% Exp3 Algorithm
\begin{algorithm}[t]
\caption{\ex{}-FLAD}
\label{alg:exp3}
\small
\begin{algorithmic}[1]

% =====================
% Begin ACL version
% =====================

% \Require $\mathcal{D}_{\mathcal{A}},\mathcal{D}_{\mathcal{T}}$: Auxiliary and target datasets
% \Require $\mathit{f}_{\theta}$: Parameterized model
% \Require $G$: Gradient accumulation steps

% \State \textbf{Initialize}: $K=\vert \mathcal{A} \vert$;\quad$\mathcal{E}_{0} = \frac{1}{K}$;
% \Statex \quad$\forall a\in \mathcal{A}: \nabla_{a} = 0$, $\hat{R}_{a} = 1$

% \For{$t=1, 2, \dots, N$}
    
%     \State $\mathcal{E}_{t} = \min \Bigl\{\frac{1}{K}, \sqrt{\frac{\ln K}{K t}} \Bigr\}$
    
%     \State $\forall a\in\mathcal{A}: p(a) \gets (1-K\mathcal{E}_{t})\frac{\exp(\mathcal{E}_{t-1}\hat{R}_{a})}{\sum_{a^{\prime}} \exp(\mathcal{E}_{t-1}R_{a^{\prime}})}+\mathcal{E}_{t}$
    
%     \State Sample dataset $a\sim p(\mathcal{A})$ and batch $\{\mathbf{x},\mathbf{y}\} \sim \mathcal{D}_{a}$
    
%     \State $\nabla_{a}$ $\gets$ $\nabla_{a}+\nabla_{\theta}\mathcal{L}(\mathit{f}_{\theta},\mathbf{x},\mathbf{y})$%\Comment{Update auxiliary dataset gradient}
    
%     \If{$t\ (\mathrm{mod}\ G) \equiv 0$}
        
%         \State $\nabla_{\mathcal{T}}\gets\nabla_{\theta}\mathcal{L}(\mathit{f}_{\theta},\mathcal{D}_{\mathcal{T}})$ %\Comment{Calculate target dataset gradient}
        
%         \State Update model parameters w.r.t.$\nabla_{\mathcal{T}}+\sum_{a}\nabla_{a}$
        
%         \ForAll{$\{ a \in\mathcal{A} \vert \nabla_{a} \neq 0\}$}
        
%             \State $\hat{R}_{a} \gets \hat{R}_{a} + \frac{R_{a,t}}{p(a)}$
            
%             \State $\nabla_{a}\gets 0$ %\Comment{Reset gradients}
%         \EndFor
%     \EndIf
% \EndFor
% =====================
% END ACL / BEGIN ICML
% =====================

\REQUIRE $\mathcal{D}_{\mathcal{A}},\mathcal{D}_{\mathcal{T}}$: Auxiliary and target datasets
\REQUIRE $\mathit{f}_{\theta}$: Parameterized model
\REQUIRE $G$: Gradient accumulation steps

\STATE \textbf{Initialize}: $K=\vert \mathcal{A} \vert$;\quad$\mathcal{E}_{0} = \frac{1}{K}$;
\STATEX \quad$\forall a\in \mathcal{A}: \nabla_{a} = 0$, $\hat{R}_{a} = 1$

\FOR{$t=1, 2, \dots, N$}
    
    \STATE $\mathcal{E}_{t} = \min \Bigl\{\frac{1}{K}, \sqrt{\frac{\ln K}{K \cdot t}} \Bigr\}$ %\COMMENT{Eq. \ref{eq:exploration_rate}}
    
    \STATE $\forall a\in\mathcal{A}: p(a) \gets (1-K\mathcal{E}_{t})\frac{\exp(\mathcal{E}_{t-1}\hat{R}_{a})}{\sum_{a^{\prime}} \exp(\mathcal{E}_{t-1}R_{a^{\prime}})}+\mathcal{E}_{t}$ %\COMMENT{Eq. \ref{eq:our_exp3_sampling}}
    
    \STATE Sample $a\sim p(\mathcal{A})$ and batch $\{\mathbf{x},\mathbf{y}\} \sim \mathcal{D}_{a}$
    
    \STATE $\nabla_{a}$ $\gets$ $\nabla_{a}+\nabla_{\theta}\mathcal{L}(\mathit{f}_{\theta},\mathbf{x},\mathbf{y})$%\Comment{Update auxiliary dataset gradient}
    
    \IF{$t\ (\mathrm{mod}\ G) \equiv 0$}
        
        \STATE $\nabla_{\mathcal{T}}\gets\nabla_{\theta}\mathcal{L}(\mathit{f}_{\theta},\mathcal{D}_{\mathcal{T}})$ %\Comment{Calculate target dataset gradient}
        
        \STATE Update model parameters w.r.t.$\nabla_{\mathcal{T}}+\sum_{a}\nabla_{a}$
        
        \FORALL{$\{ a \in\mathcal{A} \vert \nabla_{a} \neq 0\}$}
        
            \STATE $\hat{R}_{a} \gets \hat{R}_{a} + \frac{R_{a,t}}{p(a)}$
            
            \STATE $\nabla_{a}\gets 0$ %\Comment{Reset gradients}
        \ENDFOR
    \ENDIF
\ENDFOR

% =====================
% END ICML version
% =====================

\end{algorithmic}
\end{algorithm}


% UCB1 Algorithm
\begin{algorithm}[t]
\caption{\ucb{}-FLAD}
\label{alg:ucb1}
\small
\begin{algorithmic}[1]

% =====================
% Begin ACL version
% =====================

% \Require $\mathcal{D}_{\mathcal{A}},\mathcal{D}_{\mathcal{T}}$: Auxiliary and target datasets
% \Require $\mathit{f}_{\theta}$: Parameterized model
% \Require $G$: Gradient accumulation steps
% \Require $\beta$: Smoothing factor

% \State \textbf{Initialize}:
% \Statex $\forall a\in\mathcal{A}: n_{a} = 1$,
% \Statex \qquad\qquad$\hat{R}_{a} = \cos(\nabla_{\theta}\mathcal{L}(\mathit{f}_{\theta},\mathcal{D}_{\mathcal{T}}),\nabla_{\theta}\mathcal{L}(\mathit{f}_{\theta},\mathcal{D}_{a}))$ 

% \For{$t=1, 2, \dots, N$}
    
%     \State $a^{*} = \argmax_{a\in\mathcal{A}}\;\hat{R}_{a} + \sqrt{\frac{2\ln{t}}{n_{a}}}$

%     \State Sample batch $\{\mathbf{x},\mathbf{y}\}\sim \mathcal{D}_{a^{*}}$

%     \State $\nabla_{a^{*}}$ $\gets$ $\nabla_{a^{*}}+\nabla_{\theta}\mathcal{L}(\mathit{f}_{\theta},\mathbf{x},\mathbf{y})$%\Comment{Update auxiliary dataset gradient}

%     \State $n_{a^{*}} \gets n_{a^{*}} + 1$

%     \If{$t\ (\mathrm{mod}\ G) \equiv 0$}

%         \State $\nabla_{\mathcal{T}}\gets\nabla_{\theta}\mathcal{L}(\mathit{f}_{\theta},\mathcal{D}_{\mathcal{T}})$ %\Comment{Calculate target dataset gradient}

%         \State Update model parameters w.r.t. $\nabla_{\mathcal{T}}+\sum_{a}\nabla_{a}$

%         \ForAll{$\{ a \in\mathcal{A} \vert \nabla_{a} \neq 0\}$}

%             \State $\hat{R}_{a} \gets (1-\beta)\hat{R}_{a} + \beta R_{a,t}$

%             \State $\nabla_{a}\gets 0$ %\Comment{Reset gradients}
            
%         \EndFor

%     \EndIf

% \EndFor

% =====================
% END ACL / BEGIN ICML
% =====================

\REQUIRE $\mathcal{D}_{\mathcal{A}},\mathcal{D}_{\mathcal{T}}$: Auxiliary and target datasets
\REQUIRE $\mathit{f}_{\theta}$: Parameterized model
\REQUIRE $G$: Gradient accumulation steps
\REQUIRE $\beta$: Smoothing factor

\STATE \textbf{Initialize}:
\STATEX $\forall a\in\mathcal{A}: n_{a} = 1$,
\STATEX \qquad\qquad$\hat{R}_{a} = \cos(\nabla_{\theta}\mathcal{L}(\mathit{f}_{\theta},\mathcal{D}_{\mathcal{T}}),\nabla_{\theta}\mathcal{L}(\mathit{f}_{\theta},\mathcal{D}_{a}))$ 

\FOR{$t=1, 2, \dots, N$}
    
    \STATE $a^{*} = \argmax_{a\in\mathcal{A}}\;\hat{R}_{a} + \sqrt{\frac{2\ln{t}}{n_{a}}}$

    \STATE Sample batch $\{\mathbf{x},\mathbf{y}\}\sim \mathcal{D}_{a^{*}}$

    \STATE $\nabla_{a^{*}}$ $\gets$ $\nabla_{a^{*}}+\nabla_{\theta}\mathcal{L}(\mathit{f}_{\theta},\mathbf{x},\mathbf{y})$%\Comment{Update auxiliary dataset gradient}

    \STATE $n_{a^{*}} \gets n_{a^{*}} + 1$

    \IF{$t\ (\mathrm{mod}\ G) \equiv 0$}
        \STATE $\nabla_{\mathcal{T}}\gets\nabla_{\theta}\mathcal{L}(\mathit{f}_{\theta},\mathcal{D}_{\mathcal{T}})$ %\Comment{Calculate target dataset gradient}
        \STATE Update model parameters w.r.t. $\nabla_{\mathcal{T}}+\sum_{a}\nabla_{a}$
        \FORALL{$\{ a \in\mathcal{A} \vert \nabla_{a} \neq 0\}$}
            \STATE $\hat{R}_{a} \gets (1-\beta)\hat{R}_{a} + \beta R_{a,t}$
            \STATE $\nabla_{a}\gets 0$ %\Comment{Reset gradients}
        \ENDFOR
    \ENDIF
\ENDFOR

% =====================
% END ICML version
% =====================

\end{algorithmic}
\end{algorithm}

% Allowing for negative reward
In our setting, negative rewards are possible due to the use of cosine similarity to measure gradient alignment in our reward function.
Along with our first assumption (harmful auxiliary data will be harmful throughout training), we believe this leads to reasonable actions in the learner. In the original \ex{} algorithm, the use of an importance-weighted estimated reward compensates the rewards of actions that are less likely to be chosen if their empirical reward is unexpectedly high. In our variation, we allow for a similar phenomenon to occur when an empirical reward is expectedly low. Specifically, if the probability of the selected dataset $a$ is low and the learner receives a negative reward, then our learner was correct in its estimation of a low reward, further decreasing the importance of seeing dataset $a$ in the future. In the original \ex{} the learner can push up on the probability of ``good'' arms, and in our variation we give it the ability to push down on the probability of ``bad'' datasets as well.

% Our adaptation of the Exp3 algorithm
\paragraph{\ex{}-FLAD Algorithm}
% Our variation of \ex{} follows the original \ex{} closely.
On each turn, the learner first computes the current exploration rate as in Equation~\ref{eq:exploration_rate}. Then, the learner samples an auxiliary dataset from the distribution defined by Equation~\ref{eq:our_exp3_sampling}. Next, the learner samples a batch from the chosen auxiliary dataset and calculates the gradient. This process repeats $G$ times, at which point the learner calculates the gradient of the target dataset and updates the model with all auxiliary and target gradients that have accumulated. Finally, the importance-weighted reward for each auxiliary batch is calculated as in Equation~\ref{eq:importance_weighted_reward}, using our cosine reward defined in equation \ref{eq:reward}. See Algorithm~\ref{alg:exp3} for pseudo-code.
% computational complexity
% space complexity

\subsection{\ucb{} for FLAD}
While the \ex{} algorithm accounts for the stochastic, non-stationary reward generating process in FLAD, \ucb{} does not. To address this in \ucb{}-FLAD, we include an exponential moving average when estimating the mean reward for a given arm.
Formally, if at turn $t$ the learner selects auxiliary dataset $\mathcal{D}_a$, then we update the estimated mean reward $\hat{R}_{a}$ as 
\begin{equation}
\label{eq:ema_smoothing}
    \hat{R}_{a} = (1-\beta)\hat{R}_{a} + \beta R_{a,t}
\end{equation}
where $\beta$ is the smoothing factor and $R_{a,t}$ is the observed reward.

Furthermore, in the original MAB setting all interactions with the environment occur online, but FLAD is a unique situation where the learner can interact with the auxiliary data prior to training.
To take advantage of this, rather than initializing estimated rewards with a single mini-batch, we propose to initialize them with larger data quantities to improve the approximation of the true dataset gradients (line 1 of Algorithm~\ref{alg:ucb1}).
This is done for each auxiliary dataset by calculating the gradient $\nabla_{a}=\nabla_{\theta}\mathcal{L}(\mathit{f}_{\theta},\mathbf{x},\mathbf{y})$, where the number of samples in $\{\mathbf{x},\mathbf{y}\}$ is significantly larger than a mini-batch, and can be up to the size of the full dataset.

% Our adapation of the UCB1 algorithm
\paragraph{\ucb{}-FLAD Algorithm}
% Our variation on the \ucb{} algorithm mostly follows the original.
First, the estimated rewards $\hat{R}_{a}$ are initialized by approximating the gradient from each dataset as described in the previous paragraph. Then at each turn the learner computes the upper confidence bounds for each auxiliary dataset as in Equation~\ref{eq:ucb}. Next, the learner samples a batch of data from the dataset with highest upper confidence bound and calculates the gradient.
% If there is a tie, the learner selects randomly from those that are tied.
This process repeats $G$ times, at which point the learner calculates the gradient of the target dataset and updates the model with all auxiliary and target gradients that have accumulated. Finally, the smoothed estimated mean reward is calculated using Equation~\ref{eq:ema_smoothing}. See Algorithm~\ref{alg:ucb1} for pseudo-code.

% Non-stationary MAB
% Methods for non-stationary MAB are generally based on the UCB1 algorithm, including Discounted UCB \citep{kocsis2006discounted}, Sliding-Window UCB \citep{garivier2008upper}, and Adjusted UCB \citep{bouneffouf2016multi}.

% computational complexity
% space complexity

\section{Experimental Setup}
% To study our proposed algorithms, we run extensive experimentation.

\paragraph{Models}
% (1) What models do we use?
For our experiments, we utilize encoder-decoder models from the T5 family of pre-trained models \cite{raffel2020t5}. Specifically, we experiment with LM-adapted T5 (T5-LM) and T0. The T5-LM model, from \citet{lester-etal-2021-power}, further trains the T5.1.1 model for 100,000 steps (corresponding to 100B tokens) from the C4 dataset on prefix language modelling objective (i.e.\ predicting the following text based on a prefix). The T0 model is initialized from the T5-LM model and further trained on a multitask mixture of prompted datasets as described by \citet{sanh2022multitask}.
We repeat each experiment with the T5-LM XL (hereafter T5-XL) and T0-3B models (both using the same architecture with 2.85 billion parameters) as starting checkpoints (from Hugging Face Transformers~\citep{wolf-etal-2020-transformers}).
% We use the base-sized T5-LM (248 million parameters), XL-sized T5-LM, and T0-3B models (both 2.85 billion parameters).
% We repeat each experiment twice by initializing from T5-XL and T0-3B checkpoints from Hugging Face Transformers \citep{wolf-etal-2020-transformers}.

% (2) What target tasks do we use?
\paragraph{Target Datasets}
We obtain all datasets from Hugging Face Datasets\footnote{https://huggingface.co/datasets}, and cast them to the text-to-text format by applying prompt templates from the Public Pool of Prompts (P3) \citep{bach-etal-2022-promptsource} that was used to train T0.
To evaluate our few-shot methods, we utilize the same held-out datasets as T0, which cover four distinct tasks: \textbf{sentence completion} (COPA~\citep{gordon-etal-2012-semeval}, HellaSwag~\citep{zellers-etal-2019-hellaswag}, Story Cloze~\citep{sharma-etal-2018-tackling}), \textbf{natural language inference} (ANLI~\citep{nie-etal-2020-adversarial},~CB \citep{Marneffe2019TheCI}, RTE~\citep{10.1007/11736790_9}), \textbf{coreference resolution} (WSC~\citep{levesque2012winograd}, Winogrande~\citep{winogrande}), and \textbf{word sense disambiguation} (WiC~\citep{pilehvar-camacho-collados-2019-wic}). For each dataset, we randomly sample five few-shot splits from their training data, containing the same number of training examples as previous works, between 20 to 70 \citep{NEURIPS2020_1457c0d6,liu2020tfew}. We further divide each split into equal training and validation partitions. Only ANLI datasets have a publicly available test set, so for all other datasets we evaluate models on the validation set (not utilized for few-shot training or validation). For all datasets, we report the mean and standard deviation of accuracy across splits.

% (3) How do we determine the auxiliary tasks?
\paragraph{Auxiliary Datasets}
We evaluate the performance of our methods using two sets of auxiliary data and never include any of the target datasets as part of auxiliary data.
First, we use the collection of datasets used to train T0 (henceforth referred to as T0Mix), including 35 unique datasets covering the tasks of question answering, sentiment analysis, topic classification, summarization, paraphrase detection and structure-to-text.
Second, we utilize all datasets in P3 (which forms a superset of T0Mix) and, to prevent data leakage, filter out those datasets that overlap with any target dataset, leading to 260 available datasets. For each auxiliary dataset, we use at most 10,000 of the dataset's examples.

\paragraph{Training Details}
% Learning rate? 1e-4, 3e-4
% Value of G? 4 or 16 or comparably, batch sizes of 32 and 128
For direct fine-tuning, we use learning rates in $\{1e$-$4, 3e$-$4\}$ and select the best model based on validation score. For all other methods, we always use a learning rate of $1e-4$. For direct fine-tuning and FLAD baselines, we use batch sizes in $\{32,128\}$. For \ex{}-FLAD and \ucb{}-FLAD we use mini-batches of 8 samples, and let $G$ be in $\{4,16\}$ to match the batch size of other methods. For all experiments we use the Adafactor optimizer~\citep{pmlr-v80-shazeer18a} and validation-based early stopping.

% (8) Shared training details for mixed training, exp3, and ucb1
% Details: preliminary training with t5-base (not lm adapted) on subset of tasks gave scores of 0.4427, 0.4447, 0.4545, 0.4562 for full model, encoder, decoder, lm head, respectively.
In preliminary experiments we consider rewards using gradients from various model partitions: the full model weights, encoder-only weights, decoder-only weights, and the weights of the output vocabulary matrix (language modelling head). We find that using the parameters from the language modelling head provides the best performance and contains only 2.3\% of the full model parameters, significantly reducing memory consumption.
% (7) Training details specific to Exp3
% (8) Training details specific to UCB1
For \ucb{}-FLAD we found the smoothing factor $\beta=0.9$ to work well in preliminary experiments and initialize auxiliary dataset gradient alignment using 1,000 samples.

% base model has hidden size 768, and vocab of 32128 = 24,674,304 parameters in LM head
% 3b model has hidden size 2048, and vocab of 32128 = 65,798,144 parameters in LM head

More implementation details, including hyperparameters, can be found in Appendix~\ref{sec:implementation_details}
\section{Findings and Analyses}

% With T5-Base model
% \begin{table*}[t]
% \caption{Averaged results of FLAD-based methods on T0 evaluation set, reported as mean and standard deviation.}
% \label{tab:main_result}
% \vskip 0.15in
% \begin{center}
% % \begin{small}
% \begin{sc}
% \begin{tabular}{ l | c | c | c | c | c | c }
%     \toprule
%      \multicolumn{1}{r|}{Model} & \multicolumn{2}{c|}{T5-Base} & \multicolumn{2}{c|}{T5-XL} & \multicolumn{2}{c}{T0-3B} \\
%      \multicolumn{1}{r|}{Auxiliary Data} & T0Mix & P3 & T0Mix & P3 & T0Mix & P3\\         
%      \midrule
%     Direct Fine-Tuning &\multicolumn{2}{c|}{45.88\textsubscript{4.30}} & \multicolumn{2}{c|}{52.82\textsubscript{3.34}} & \multicolumn{2}{c}{56.44\textsubscript{4.70}} \\
%     % \hline
%     Exploration-only & 46.93\textsubscript{3.14}&45.46\textsubscript{5.29} & 59.18\textsubscript{5.52} & 60.64\textsubscript{4.92} & 61.17\textsubscript{3.30} & 62.77\textsubscript{4.83} \\
%     % \hline
%     Exploitation-only & 47.16\textsubscript{3.15} & 46.11\textsubscript{4.27} & 59.79\textsubscript{5.63} & 60.49\textsubscript{5.01} & 60.87\textsubscript{3.35} & 62.87\textsubscript{3.69} \\
%     % \hline
%     \ex{}-FLAD & 48.21\textsubscript{4.55} & 47.44\textsubscript{4.37} & 61.50\textsubscript{4.25} & 64.07\textsubscript{4.81} & 62.87\textsubscript{3.85} & \underline{65.98}\textsubscript{3.20} \\
%     % \hline
%     \ucb{}-FLAD & 48.26\textsubscript{3.53} & 48.41\textsubscript{3.67} & 62.01\textsubscript{3.89} & \underline{65.52}\textsubscript{3.86} & 62.89\textsubscript{3.68} & \textbf{66.29}\textsubscript{3.29} \\
%     \bottomrule
% \end{tabular}
% \end{sc}
% % \end{small}
% \end{center}
% \vskip -0.1in
% \end{table*}

% Only 3B models
\begin{table}[t]
\caption{Results of target-only fine-tuning and FLAD-based methods on our target datasets (i.e.\ the held out datasets from T0). Reported scores are mean accuracy and standard deviation over the full evaluation set.}
\label{tab:main_result}
\vskip 0.15in
\begin{center}
\begin{small}
% \begin{sc}
\begin{tabular}{ l | c | c | c | c }
    \toprule
     \multicolumn{1}{r|}{\textsc{Base Model}} & \multicolumn{2}{c|}{\textsc{T5-XL}} & \multicolumn{2}{c}{\textsc{T0-3B}} \\
     \multicolumn{1}{r|}{\textsc{Aux. Data}} & \textsc{T0Mix} & \textsc{P3} & \textsc{T0Mix} & \textsc{P3}
     \\
     \midrule
    Target-Only & \multicolumn{2}{c|}{52.82\textsubscript{3.34}} & \multicolumn{2}{c}{56.44\textsubscript{4.70}} \\
    % \hline
    Explore-Only & 59.18\textsubscript{5.52} & 60.64\textsubscript{4.92} & 61.17\textsubscript{3.30} & 62.77\textsubscript{4.83} \\
    % \hline
    Exploit-Only & 59.79\textsubscript{5.63} & 60.49\textsubscript{5.01} & 60.87\textsubscript{3.35} & 62.87\textsubscript{3.69} \\
    % \hline
    \ex{}-FLAD & 61.50\textsubscript{4.25} & 64.07\textsubscript{4.81} & 62.87\textsubscript{3.85} & \underline{65.98}\textsubscript{3.20} \\
    % \hline
    \ucb{}-FLAD & 62.01\textsubscript{3.89} & \underline{65.52}\textsubscript{3.86} & 62.89\textsubscript{3.68} & \textbf{66.29}\textsubscript{3.29} \\
    \bottomrule
\end{tabular}
% \end{sc}
\end{small}
\end{center}
\vskip -0.1in
\end{table}

% \begin{figure}[t]
%     \centering
%     \includegraphics[width=\columnwidth]{images/UCB1_dynamics.png}
%     \vspace{-5mm}
%     \caption{\textbf{Training Dynamics of UCB1}, a case study using WSC as target dataset and T0Mix as auxiliary data. Colored lines represent a sample of auxiliary datasets with qualitatively interesting behavior; the remaining datasets are shown in grey. Shown on top, middle, and bottom are the calculated upper confidence index, the gradient similarity, and the percent of total data that has been sampled from each dataset respectively.}
%     \label{fig:ucb1_dynamics}
% \end{figure}

Now that we have designed and described the \ex{}-FLAD and \ucb{}-FLAD algorithms, we first compare their performance with other FLAD methods. Aggregate scores are shown in Table~\ref{tab:main_result} with detailed results in Table~\ref{tab:detailed_results} of the Appendix. Then, we compare our methods with strong few-shot methods and show the full results in Figure~\ref{fig:comparison}.
% Below may be unnecessary
% We discuss these results in detail in the following paragraphs.


% \paragraph{The Power of FLAD}
\paragraph{FLAD Helps Models Generalize}
First, we compare our proposed methods with target-only fine-tuning (i.e.\ without using auxiliary data) as well as two baseline FLAD settings (explore-only and exploit-only). \textit{Explore-only} is equivalent to multi-task training where auxiliary data is uniformly sampled, i.e.\ continuously exploring auxiliary data and never exploiting knowledge of its relation to the target data. \textit{Exploit-only} computes gradient alignment prior to training, as in \ucb{}, followed by multi-task training where the sampling probabilities are defined by a Gibbs distribution over similarities (similar to that in \ex{}) which results in always exploiting and never exploring. For both FLAD baselines we utilize mixed-dataset batches and a target dataset mixing ratio with the possible values of $\{1,5,10\}$ times the greatest auxiliary sampling probability.

Table \ref{tab:main_result} shows that \textit{all} FLAD methods provide significant improvement in few-shot generalization over target-only fine-tuning, proving the utility of FLAD methods. We find that even the naive mixing approach, \textit{explore-only}, improves performance by 4.8 - 8\%, with the remaining approaches improving accuracy by up to 12.7\%.
Furthermore we find that, given identical training data, all single-stage FLAD methods on T5-XL + T0Mix (left column of Table~\ref{tab:main_result}) improve over the multi-task trained T0 with target-only fine-tuning (between 2.7 - 5.6\% improvement), demonstrating how simultaneous training on auxiliary and target data, as in FLAD methods, can help models generalize in few-shot settings.

\paragraph{The Importance of Exploration \textit{and} Exploitation}
% Although naive FLAD methods improve over target-only training, the improvements quickly taper off as more auxiliary data is added.
% Exploration and exploitation together leads to continued performance improvement.
Our experiments show that \ex{}-FLAD and \ucb{}-FLAD consistently outperform the explore-only and exploit-only FLAD methods across both models and auxiliary datasets. Additionally, Table~\ref{tab:detailed_results} of the appendix shows that our methods surpass the baselines in \textit{every} evaluation task.

Furthermore, when comparing the ability of FLAD methods to leverage additional auxiliary data (i.e.\ going from T0Mix to all of P3), we find that the improvement for explore- and exploit-only methods is minimal with only 0.7-2\% improvement. On the other hand, \ex{}-FLAD and \ucb{}-FLAD show a notable improvement of 2.6-3.5\%, emphasizing the importance of both exploration \textit{and} exploitation, particularly when dealing with large collections of auxiliary data.


\begin{figure}[t]
    \centering
    \includegraphics[width=0.95\columnwidth]{images/sampling_frequency.pdf}
    \vspace{-2mm}
    \caption{\textbf{Empirical Sampling Distributions of FLAD Methods}. A case study on RTE as the target dataset and T0Mixture as auxiliary data.
    % , where \ex{}-FLAD (top) and \ucb{}-FLAD (bottom) outperform baselines.
    Top-4 most sampled auxiliary datasets are highlighted with color. \ucb{}-FLAD forms a distribution with two clear peaks, while \ex{}-FLAD forms a more flat distribution.}
    \label{fig:sampling_frequency}
\end{figure}

\begin{figure*}
    \centering
    \includegraphics[width=0.95\textwidth]{images/flad_comparison.pdf}
    \vspace{-2mm}
    \caption{\textbf{Comparison of FLAD methods trained on P3 with previous few-shot methods}. We calculate all T-Few scores on our data splits using code from~\citet{liu2020tfew}. DEFT-Few scores are taken from~\citet{Ivison2022DEFT}. GPT-3 scores are taken from~\citet{NEURIPS2020_1457c0d6} and utilize few-shot in-context learning. All models utilize the same number of few-shot examples and (other than GPT-3) have 3B parameters.}
    \label{fig:comparison}
\end{figure*}

\paragraph{Comparing \ex{}-FLAD and \ucb{}-FLAD}
In Section~\ref{sec:modelling_assumptions} we argue that adversarial MAB is an appropriate model of the FLAD setting, but our results empirically show that \ex{}-FLAD is slightly outperformed by \ucb{}-FLAD. In this section we analyze each method's performance and training dynamics to understand this phenomenon.

To better understand the training dynamics, we perform a case study on T5-XL with T0Mix as the auxiliary data, with full details and figures in Appendix~\ref{sec:training_dynamics}.
% UCB1 gives a peaky empirical sampling distribution
First we look at RTE, where \ucb{}-FLAD outperforms \ex{}-FLAD. We calculate the empirical distribution of samples seen from each auxiliary dataset, shown in Figure~\ref{fig:sampling_frequency}, and find that \ex{}-FLAD samples nearly uniformly from all datasets while \ucb{}-FLAD forms a bimodal sampling distribution with 2 very clear peaks.
The lack of peakiness in the \ex{}-FLAD distribution is counterintuitive, as we do find that it achieves separation between auxiliary tasks in the cumulative estimated reward (as shown in Figure~\ref{fig:ex_rte}), but this does not lead to separation in the sampling probability space. 
Additionally we find that even on COPA, where \ex{}-FLAD outperforms \ucb{}-FLAD, \ex{}-FLAD still achieves good separation between cumulative estimated rewards, but has a monomodal sampling distribution, while \ucb{}-FLAD does not have as clear of a bimodal distribution as in RTE.

% What causes this peakiness?
The difference in empirical sampling distributions is likely due to the difference between the greedy policy of \ucb{}-FLAD and the stochastic policy of \ex{}-fLAD. Empirically, we find that \ex{}-FLAD very rarely assigns an auxiliary dataset a probability $<1$\%, leading to many ``bad'' batches over the course of thousands of turns. On the other hand, the optimistic policy of \ucb{}-FLAD spends much less time exploring and will sample ``bad'' batches much less frequently.

% While we find that Exp3 is able to very quickly adjust the expected rewards for a dataset, it turns out this is overruled by the non-zero sampling probability it assigns to ``bad'' datasets.
% EXP3 assigns too much weight to exploring (non-zero probability even for really low cumulative reward), and never separates out the good datasets, partly exaggerated by the fact that it never removes datasets from consideration. Turns out our setting does not have adversarial changes in "best" auxiliary datasets. 

\paragraph{Empirical Findings on Gradient Alignment as a Reward}
We find gradient alignment to be a very noisy signal (see Appendix~\ref{sec:training_dynamics}). For \ex{}-FLAD, this may be harm the formation of a policy because the adversary is frequently changing which auxiliary dataset is best, forcing it to make large changes to the sampling distribution. On the other hand, \ucb{}-FLAD smooths the reward signal with an exponential moving average, leading to the model sampling multiple batches from an auxiliary dataset before adjusting the confidence bounds. By sampling multiple batches, \ucb{}-FLAD gets a better approximation for the auxiliary dataset reward.

Furthermore, we find that the rewards are mostly stationary (although highly stochastic) after an initial break-in period of $\sim$100 gradient updates, tending towards 0.
% NOTE: Gradients tending towards 0 may mean that our assumption (negative rewards for EXP3 is okay) is actually wrong, and is hurting the models performance. Unclear whether that is the case though

\paragraph{How can \ex{}-FLAD and \ucb{}-FLAD be improved?}
It is clear that FLAD is not a very strong adversary and that gradient alignments maintain a weak form of stationarity. Taking these findings into account, future methods can utilize this information to improve. First, we show that although gradient alignments tend towards 0, there is useful information in the noisy signal, as proven empirically, which models must try to utilize. Next, we believe that smoothing the reward function can significantly help future methods to make confident decisions. \alon{Is this section meaningful? maybe save the space for related works or discussion instead?} % Not super meaningful but not a waste - may be better for the conclusion


% How much do the specific dataset orderings vary across seeds? Fairly stable, use WSC as example, amazon_polarity is ALWAYS top and next few change exact ordering, but generally are top
% Similar analysis for Exp3, cumulative estimated reward for most similar datasets

% (6) Opposite trend for base vs. large models for increasing K. Base models get worse overall when switching from T0Mixture to P3. Large models get better. \alon{Only if including experiments with base-sized model}


% \paragraph{Training Dynamics of \ucb{}-FLAD}
% \alon{We do a small-scale analysis on the training dynamics of \ucb{}-FLAD to better understand it's performance, and how to improve on this algorithm in the future. Gradients jump from negative to positive.}
% (4) Analysis of learner (how do similarities change over time, compared with static policy)
% Interesting analysis would be to look at the distribution of batch similarities, see just how noisy our reward signal is (spoiler, VERY NOISY)
% for Exp3, look at instantaneous sampling distribution (defined at each time step), or at cumulative estimated (cumulative importance weighted) reward to see what exp3 believes it's reward has been
% for ucb1 look at upper confidence index and empirical dataset proportion
% To make figures more readable, we only show auxiliary datasets which were in the top/bottom k at the beginning and end of training

\paragraph{FLAD Provides Robustness Compared with Non-FLAD Methods}
In this section, we compare the performance of our methods trained on P3 with competitive few-shot methods: T-Few, DEFT-Few, and GPT-3.
T-Few~\citep{liu2020tfew} is a variant of the T0-3B model that multi-task pre-trains parameter-efficient IA$^{3}$ modules followed by target-only fine-tuning the (IA)$^{3}$ modules.
DEFT-Few~\citep{Ivison2022DEFT} is a variant of the T5-XL model that has been multi-task trained using retrieved auxiliary data similar to the target dataset. Using 1000 unlabeled target dataset samples, DEFT first multi-task trains a T5-XL model on the 500 nearest neighbor samples from P3. DEFT then involves target-only fine-tuning with the (IA)$^{3}$ modules from~\citet{liu2020tfew}.
Finally, we also compare against the 175 billion parameter variant of GPT-3~\citep{NEURIPS2020_1457c0d6}, which utilizes in-context learning with the same number of few-shot examples as all other models.

We find that, on average, T0 models trained using our FLAD-based methods outperform all other methods, and to the best of our knowledge, our methods lead to the first 3 billion parameter models that outperform GPT-3 on this dataset mixture (previous smallest models have 11 billion parameters). Additionally, the T5-XL models trained with our FLAD-based methods are only outperformed by GPT-3, which has $62.5$ times more parameters. % Don't these two sentences contradict each other?
Additionally, we find that our FLAD-based methods provide robust performance across datasets, achieving best or second-best performance on $9/11$ datasets, only perfomorning worst on COPA.

\section{Related Work}


% Works that combined MAB into supervised learning
Prior works have utilized MAB as a method of improving supervised learning. For example,~\citet{Pasunuru2020DORBDO} use MAB to optimize multiple rewards for language generation, and~\citet{Guo2018DynamicMM} use it to select auxiliary tasks to improve sentence simplification.
% Works that combine auxiliary data into training
Some early works found success when combining auxiliary data into target-aware training such as~\citep{10.1145/1015330.1015436}. \citet{chen2022weighted} perform target-aware training, but assume access to 10,000 target samples (not few-shot) and only utilize 1 auxiliary dataset. Yet other works~\citep{Lin2022UnsupervisedCG,Ivison2022DEFT} incorporate auxiliary data by retrieving samples similar to the target dataset, but assume access to 1,000 unlabeled target examples to create a quality representation for retrieval. More similar to our work are~\citet{Verboven2022, https://doi.org/10.48550/arxiv.1812.02224}, which adaptively weight auxiliary losses based on gradient alignment, but they both evaluate on settings with only 2 auxiliary datasets.



% (1) Multi-armed bandits uses
% \citet{graves2017automated} use multi-armed bandits to automate curriculum learning with the \ex{} algorithm. They propose multiple interesting gradient-based reward functions which could be used to replace our similarity reward, but which require additional computation.

% (2) Prior works that consider training with auxiliary data

% \citep{Verboven2022} - adaptively weight loss functions from auxiliary tasks based on mini-batch similarities similar to our work. Only evaluated in toy settings on synthetic data and using a single auxiliary task at a time.

% \citep{https://doi.org/10.48550/arxiv.1812.02224} - Use cosine similarity between multiple objectives to scale the loss function, only 2 tasks

% \citep{NEURIPS2019_0e900ad8} - using auxiliary tasks in an RL setting, they want to find tasks whose gradients point in the same direction as target task

% \citep{deryAANG2022} - Similar to our work. They focus on using unsupervised auxiliary data with many possible self-supervised auxiliary objectives.

% \citep{chen2022weighted} - present the Target-Aware Weighted Training algorithm, tries to learn a representation of the target task. Crucially, they assume access to a large enough quantity of target data ($\sim$10K samples) to allow for such a representation to be learned, and they do not consider settings with more than 1 auxiliary dataset.

% Methods such as ReCross \cite{Lin2022UnsupervisedCG} and DEFT \cite{Ivison2022DEFT} aim to incorporate auxiliary datasets during training for downstream tasks, but also assume access to up to 1000 unlabeled target examples.

% (3) Curriculum Learning
% \citep{wang2021survey} - a survey of curriculum learning

% (4) Multi-task and meta-learning

% Transfer learning falls under the static distribution dilemma
% Meta-learning focuses on a general learning algorithm that will work well for any 
% Multi-task learning gives equal weight to all tasks during MTL training, missing out on taking advantage of end-task specific knowledge

% Recent works on the FLAD problem have utilized methods from transfer learning \citep{vu-etal-2020-exploring, pruksachatkun-etal-2020-intermediate, feta_albalak}(\alon{update FETA with EMNLP proceeding}), meta-learning \citep{10.5555/296635, bansal-etal-2020-self}, and multi-task learning \citep{Wu2020Understanding, aghajanyan-etal-2021-muppet, sanh2022multitask, Verboven2022}.
% % Transfer Learning
% Prior works in transfer learning have proposed methods of determining which datasets will provide positive transfer while avoiding negative transfer, but these methods generally focus only on a single source of auxiliary data. Additionally, transfer learning methods assume that the quality of transfer is static and disregard the dynamics of training, reducing computational complexity but leading to suboptimal transfer. 
% % Meta-Learning and Multi-Task Learning
% Methods based on meta-learning and multi-task learning lead to models which can be quickly trained on many new tasks by splitting their supervised training into two phases. In the first phase, the target data is given no weight, or equal weight to the auxiliary data, and in the second phase the target data is trained on alone. This first target-agnostic training phase can lead to models which are optimal for generalizing to a large number of new tasks, but suboptimal for training on any specific task.

% \begin{itemize}
%   \item MTL doesn't consider the target task during MTL, only during fine-tuning
%   \item In-context learning with massive language models avoid the training data selection problem, but fall into a parallel problem of selecting the in-context examples to use. Additionally, many studies have still found that smaller trained models can largely outperform such generalist models.
%   \item Works on transfer learning have proposed methods of determining which datasets provide positive transfer while avoiding negative transfer, but such methods do not consider the dynamics of training and generally focus only on single source to single target transfer
%   \item Data selection and curriculum learning can be very expensive with such large quantities of auxiliary data
% \end{itemize}

% Few-shot learning is an attractive paradigm for many machine learning practitioners as it can save on cost and time for data collection and annotation. One method of performing few-shot learning is to find existing datasets for the domain, task, and language of interest. Over time, the number of annotated datasets has increased, allowing for practitioners to find more data that fits the target domain and task. Transfer learning is a field that aims to understand which auxiliary tasks/domain/languages can be used to improve model performance on the target task/domain/language, however, transfer learning generally considers the stationary problem of determining transferrability prior to training a new model. This disregards the dynamics of training entirely, leading to suboptimal performance.

% \citep{bach-etal-2022-promptsource} - promptsource

% \citep{supernaturalinstructions} - nat instruct v2

% \citep{aribandi2022ext} - ExT5

% (5) Data selection

% \citep{sorscher2022beyond} Beyond neural scaling laws, A method for dataset pruning for very large datasets
% They propose a self-supervised data pruning method which groups unlabeled data points by a clustering algorithm and determines self-supervised prototypes
% Once they have clusters, they determine that “hard” samples are those farthest from the cluster centroid, and easier samples are those which are closer

% Prioritized training on points that are learnable, worth learning, and noy yet learnt

% Numerous recent works on instance selection \cite{Siddiqui2022MetadataAU}, which cover a few situations. Some methods focus on model pre-training \cite{Sorscher2022BeyondNS, Mindermann2021PrioritizedTO} where the idea is to select a subset of data D, where D << All-Data, and D is a better selection of samples which allows for similar or better performance to training on All-Data. When these methods are done online, this can be seen as a form of curriculum learning. If these methods are used offline, they can be very useful for gaining a better understanding of the training data. On the other hand, instance selection can be used in fine-tuning settings. Studies on instance selection during fine-tuning assume large quantities of in-domain target task-specific data. Again, when these methods are used in an online setting, they are a form of curriculum learning. However, in this scenario, when methods are used offline, they can be used to identify possibly mislabeled data, subgroups, instance difficulty, and dataset biases.

% Methods such as multi-task learning and meta-learning focus on a similar problem to our setting, but they generally perform a final fine-tuning on the target task (when data is available) which does not utilize the available auxiliary data. \cite{Dery2021ShouldWB} argue for end-task aware training, as an alternative to pre-training and fine-tuning. However, their methods focus on tasks with large unlabeled domain-specific datasets to benefit the pre-training and do not consider the few-shot setting, which has been shown to benefit from general pre-training.

% Prior works have found success in improving downstream task performance by continued pre-training simultaneous to fine-tuning. (zack lipton's recent work is an example "downstream datasets make surprisingly good pretraining corpora")

% \begin{itemize}
%     \item \cite{Pasunuru2020DORBDO} use MAB as a method of optimizing multiple rewards for language generation tasks.
%     \item \cite{Guo2018DynamicMM} use multi-armed bandits based approach for selecting from auxiliary tasks for sentence simplification task.
%     \item \cite{Sharma2017OnlineML} good example of use of bandits in RL for multi-task learning
%     \item \cite{lattimore_szepesvári_2020} Big book on bandit algorithms (suggested by Dheeraj)
%     \item \cite{wei2021non} Very interesting future approach, since similarity distributions are non-stationary and training has stochasticity, this algorithm suggests that multiple training runs can be performed which will lead to better performance and improved worst-case outcomes (suggested by Dheeraj)
%     \item \cite{simpkins2008optimal} Use in introduction when talking about exploration-exploitation tradeoff. Example of exploration-exploitation in optimal control scenario
%     \item \cite{shaham2022causes} consider the causes for negative interference between languages in multilingual translation, finding that model size, data size, and proportion of data coming from each language in the mixture. Unsurprisingly, they find that by increasing model size, a multilingual model has less negative interference because it has greater capacity for each language.
%     \item \cite{pmlr-v24-seldin12a} We take our exp3 algorithm directly from Algorithm 1 in this work. This is a slight variation of the original Exp3 algorithm \cite{auer2002nonstochastic} which varies the the exploration rate over time. Because the exploration rate never drops to 0 and we have relatively short play horizons (<10000 steps), the algorithm will sample ``bad'' actions at a rate of $\epsilon_{t}=\sqrt{\frac{\mathrm{ln}K}{Kt}}$ where $t$ is the current turn. Under a stationary distribution such continued exploration is a negative consequence, but in our scenario we do not have stationarity, therefore we must allow for such a continued exploration.
%     \item \cite{auer2002finite} introduce the UCB1 algorithm
%     \item \cite{bubeck2009} discuss the differences in exploration-exploitation strategies under 2 settings: (1) when exploration is only constrained by the number of available rounds (not necessarily known in advance), or (2) when the cumulative regret is considered and when exploitation needs to be performed at the same time.
%     \item \cite{Katharopoulos2018NotAS} Use a method based on importance sampling (related to UCB1) for solving the sample selection problem on single-task fine-tuning.
%     \item \cite{10.1145/1015330.1015436} ``This paper explores  an application in which a second, auxiliary, source of data is available drawn from a different distribution. This auxiliary data is more plentiful, but of significantly lower quality, than the training and test data.'' This work is from 2004, uses SVM, and demonstrates that using auxiliary data is not a new idea. Another work more recently, \cite{esfandiarpoor2021} also considers using labeled auxiliary data in the few-shot setting, but in a much more constrained setting where the auxiliary data are new classes
% \end{itemize}
% \section{Discussion}

\paragraph{Efficiency}
\alon{How do our algorithms do in terms of space-complexity?}
\alon{How do our algorithms do in terms of computational complexity?}

% Design choices that may be leading to suboptimal performance:
% At each turn, our algorithms choose an auxiliary dataset, calculate the gradient, and after G rounds will use that gradient to update the model. There is likely an approach to filtering out poor gradients that will improve model performance
% We consider a very simple reward function that can be computed very efficiently in an online algorithm, but previous works on multi-armed bandits for training neural networks have utilized more complicated rewards such as in \citet{graves2017automated}. It is likely that our reward function can be improved upon, especially based on our findings of the dynamics of task similarity throughout training.

% Gradient similarity as a proxy for success
% We have formulated our reward based on gradient similarity between an auxiliary dataset and the target dataset. In an extreme case, if the auxiliary data has exactly the same gradient as the target data, our algorithms would heavily prefer this data, possibly leading to the exact same training outcome as directly fine-tuning on the target data alone. This would not be an ideal situation, but we make the assumption that our auxiliary data will never have identical gradients to the target. \alon{This assumption may be too obvious, good candidate for removal.}

% Because Exp3 makes fewer assumptions on the reward generating distribution, it proves to be a more general solution to bandit algorithms, and therefore likely to not be the best policy. We find that UCB1 leads to better performance, suggesting that there may be some properties of our setting that we do not know, and with further analysis of the training dynamics, we may be able to create an algorithm with improved properties and maybe even guarantees.

% Reward function, we only consider gradient similarity, what else would be reasonable?
\section{Conclusion}
The methods proposed in this work demonstrate the effectiveness of simultaneous training on auxiliary and target datasets, as well as the importance of continuously updating beliefs by exploring \textit{and} exploiting auxiliary data.
% Promote the need to do further work in utilizing auxiliary data by 1) reducing computational complexity, 2) designing rewards better suited to this scenario, 3) designing algorithms more specialized to supervised learning

\alon{Circle back to desiderata in introduction and show that we've accomplished what we set out to do.}
\subsection{Desiderata}
% Recall our desiderata and make sure to remind readers how we will do these things
Briefly recalling our desiderata for a FLAD algorithm, we wish for our learner to: make no assumptions on the auxiliary data a-priori to training, update it's beliefs on the importance of each auxiliary dataset online, and run efficiently in both memory and computational resources. By framing FLAD in the MAB setting we maintain an importance on exploration of the auxiliary data, requiring no information a-priori on the distributions, and including the ability to update beliefs continuously. For the last point, we made design choices (calculate gradient similarities using only a small portion of model parameters) that allow for very reasonable space and time complexity.



\bibliography{custom,anthology}
\bibliographystyle{icml2023}


%%%%%%%%%%%%%%%%%%%%%%%%%%%%%%%%%%%%%%%%%%%%%%%%%%%%%%%%%%%%%%%%%%%%%%%%%%%%%%%
%%%%%%%%%%%%%%%%%%%%%%%%%%%%%%%%%%%%%%%%%%%%%%%%%%%%%%%%%%%%%%%%%%%%%%%%%%%%%%%
% APPENDIX
%%%%%%%%%%%%%%%%%%%%%%%%%%%%%%%%%%%%%%%%%%%%%%%%%%%%%%%%%%%%%%%%%%%%%%%%%%%%%%%
%%%%%%%%%%%%%%%%%%%%%%%%%%%%%%%%%%%%%%%%%%%%%%%%%%%%%%%%%%%%%%%%%%%%%%%%%%%%%%%
\newpage
\appendix

\section{Implementation Details}
\label{sec:implementation_details}
% (5) Training details specific to direct fine-tuning
% (6) Training details specific to exploration-only and exploitation-only methods
% (7) Training details specific to Exp3
% (8) Training details specific to UCB1
% Smoothing factor - initially experimented with 0,99, 0.9, 0.8, 0.5. Seemed like 0.9 was a happy medium for most target datasets.

% learning rates, gradient accumulation steps, batch sizes
\alon{Fill in hyperparameters here.}
\section{More Training Details}
For all experiments, we use validation-based early stopping, and train for a maximum of 10,000 gradient update steps. In practice, we find that early-stopping leads to significantly fewer than 10,000 updates, usually between 50-150 for direct fine-tuning, and 1-2,000 for other methods.

For the smoothing factor, $\beta$, in \ucb{}-FLAD we ran preliminary experiments using values of $\{0.99, 0.9, 0.75, 0.5\}$ and found 0.9 to work well across datasets. All reported scores use $\beta=0.9$.

In preliminary experiments we consider rewards using gradients from multiple model partitions: the full model, encoder-only, decoder-only, and language modelling head (token classifier). We find that using the parameters from the LM head provides best performance, followed by the decoder-only, encoder-only, and full model gradients. The differential from best to worst method was $\sim3\%$ relative performance. Recall that with a gradient accumulation factor of $G$, our algorithms need to store at most $G+1$ gradients at any time. So not only does using the LM head provide performance improvements, but also saves memory. For the models we use, the LM head contains only 2.3\% of the full model parameters.

\section{Which portion of the language model should be used to compare gradients?}
In our initial experiments with T5-base, we compared the use of different model partitions for gradient comparisons. We considered the use of the full model gradient, encoder-only, decoder-only, and language modelling layer. We found that using the language modelling layer only performed the best (~3\% better than using the entire model), with the decoder layers and encoder layers also outperforming the full model by 2.6\% and 0.5\%, respectively.
While these findings seems to suggest that using the language modelling layer is best, when looking at individual tasks, we find that this is not true for 3/8 tasks (story cloze, wic, and winogrande). These 3 tasks 


wic is words-in-context, using intermediate representation is probably better than just token probabilities.
Story cloze and winogrande are tasks which require commonsense knowledge.

Other tasks may be more focused on linguistic features rather than semantics. ANLI is nli, CB is also NLI, copa is selection of sentence completions (relies on some commonsense, and also linguistic features). hella swag is commonsense NLI. rte is also NLI, 


\section{Things we didn't try - rephrase this}
In this work, we define the action reward of selecting an auxiliary dataset to be the cosine similarity between the auxiliary batch gradient and the target task gradient. Using different similarity/distance measures may prove to be useful as well, such as an $L_{2}$ or $L_{1}$ distance. Similarly, gradient directions are not necessarily the best proxy for dataset similarity. Furthermore, in this work we make the assumption that dataset similarity is the objective to be optimized, but this may not be the best metric to optimize for.

\section{Full Results}
\label{sec:appendix_results}


\begin{table}
\centering
\small
\label{tab:detailed_results}
\caption{Detailed results from the main experiment including direct fine-tuning, exploration-only, exploitation-only baselines and our proposed methods, \ex{}-FLAD and \ucb{}-FLAD.}
\begin{adjustbox}{angle=-90}
\begin{tabular}{|lrr|r|r|r|r|r|r|r|r|r|r|r|r|r}\toprule
\multicolumn{3}{r|}{Dataset} &Anli-r1 &Anli-r2 &Anli-r3 &CB &COPA &HellaSwag &RTE &Story Cloze &WiC &Winogrande &WSC &Average \\
\hline
&\multicolumn{2}{r|}{Direct Fine-Tuning} &37.6 &36.2 &35.0 &83.2 &53.8 &51.0 &54.2 &75.9 &51.6 &49.6 &53.1 &52.8 \\
\multirow{8}{*}{T5-3B} &\multirow{ 4}{*}{T0Mix} &Exploration-Only &38.1 &40.3 &36.7 &88.6 &85.6 &51.2 &67.6 &88.8 &51.0 &55.5 &47.7 &59.2 \\
& &Exploitation-Only &38.8 &40.5 &38.0 &86.1 &86.0 &51.1 &69.4 &89.5 &52.8 &59.2 &46.3 &59.8 \\
& &Exp3-FLAD &40.6 &39.9 &36.9 &86.1 &\textbf{89.8} &\textbf{52.0} &76.7 &90.8 &50.5 &60.3 &52.9 &61.5 \\
& &UCB1-FLAD &41.8 &39.0 &38.0 &85.4 &87.0 &\textbf{52.0} &79.1 &91.4 &49.7 &62.7 &56.2 &62.0 \\
\cmidrule{2-15}
& \multirow{4}{*}{P3} &Exploration-Only &40.1 &37.7 &36.0 &85.4 &83.6 &\textbf{52.1} &77.3 &89.1 &51.5 &57.2 &57.1 &60.6 \\
& &Exploitation-Only &40.4 &37.2 &37.3 &87.1 &84.4 &51.0 &78.6 &90.3 &51.3 &56.2 &51.5 &60.5 \\
& &Exp3-FLAD &46.9 &38.8 &40.2 &89.6 &88.0 &51.5 &76.9 &91.2 &53.4 &66.2 &61.9 &64.1 \\
& &UCB1-FLAD &49.1 &38.8 &40.1 &88.6 &88.2 &51.6 &83.7 &90.2 &\textbf{54.3} &\textbf{68.0} &68.3 &65.5 \\
\hline
 &\multicolumn{2}{r|}{Direct Fine-Tuning} &40.9 &39.1 &37.1 &79.6 &66.4 &43.5 &67.1 &83.2 &52.5 &54.6 &56.7 &56.4 \\
\multirow{8}{*}{T0-3B} &\multirow{ 4}{*}{T0Mix} &Exploration-Only &44.4 &40.3 &37.0 &82.5 &85.6 &47.9 &77.6 &90.1 &52.1 &58.6 &56.9 &61.2 \\
& &Exploitation-Only &42.5 &39.3 &37.2 &84.3 &82.8 &48.1 &79.7 &88.8 &52.8 &57.8 &56.3 &60.9 \\
& &Exp3-FLAD &46.2 &41.5 &37.7 &83.9 &87.6 &49.4 &80.0 &90.1 &52.6 &63.4 &59.0 &62.9 \\
& &UCB1-FLAD &43.7 &40.8 &37.6 &86.1 &85.4 &48.6 &80.5 &91.3 &53.4 &63.5 &61.0 &62.9 \\
\cmidrule{2-15}
&\multirow{4}{*}{P3} &Exploration-Only &45.4 &40.3 &38.0 &82.5 &87.8 &50.6 &82.2 &88.8 &52.4 &61.8 &60.6 &62.8 \\
& &Exploitation-Only &45.5 &40.0 &38.8 &87.5 &82.2 &49.9 &79.6 &90.9 &52.2 &60.1 &64.8 &62.9 \\
& &Exp3-FLAD &\textbf{50.4} &40.0 &\textbf{41.2} &87.9 &88.4 &49.7 &\textbf{86.1} &\textbf{91.6} &52.8 &67.5 &70.4 &66.0 \\
& &UCB1-FLAD &48.2 &\textbf{41.8} &\textbf{41.2} &\textbf{90.0} &86.6 &50.0 &\textbf{86.1} &\textbf{91.5} &53.6 &65.6 &\textbf{74.6} &66.3 \\
\bottomrule
\end{tabular}
\end{adjustbox}
\end{table}

\clearpage

\section{\ex{}-FLAD and \ucb{}-FLAD Training Dynamics}
\label{sec:training_dynamics}
The following 4 pages include a case study on the training dynamics of \ex{}-FLAD and \ucb{}-FLAD when training T5-XL using T0Mix as the auxiliary data. First, we find datasets where \ex{}-FLAD and \ucb{}-FLAD improve significantly over the baseline FLAD methods, but also where either \ex{}-FLAD or \ucb{}-FLAD clearly outperforms the other. The two datasets that fulfill our interests are RTE and COPA.

We find that \ucb{}-FLAD outperforms \ex{}-FLAD on RTE, and show their respective training dynamics in Figure~\ref{fig:ucb_rte} (\ucb{}) and Figure~\ref{fig:ex_rte} (\ex{}).
We find that \ex{}-FLAD outperforms \ucb{}-FLAD on COPA, and show their respective training dynamics in Figure~\ref{fig:ucb_copa} (\ucb{}) and Figure~\ref{fig:ex_copa} (\ex{}).

We include details and takeaways in the caption for each figure. For \ex{}-FLAD figures we include charts of the cumulative estimated reward, empirical gradient alignment, instantaneous sampling distribution determined by the policy, and the empirical sampling distribution determined by the total number of samples seen per dataset as a fraction of the total samples seen. For \ucb{}-FLAD figures, we include charts of the upper confidence index, estimated gradient alignment, and the empirical sampling distribution.

\begin{figure*}
    \centering
    \includegraphics[width=0.95\textwidth]{images/UCB1_dynamics_rte.pdf}
    \caption{Training dynamics of \ucb{}-FLAD, a case study using RTE as target dataset and T0Mix as auxiliary data, where \ucb{}-FLAD outperforms. Colored lines are a sample of auxiliary datasets with interesting properties, the remaining datasets are shown in grey.}
    \label{fig:ucb_rte}
\end{figure*}

\begin{figure*}[t]
    \centering
    \includegraphics[width=0.95\textwidth]{images/EXP3_dynamics_rte.pdf}
    \caption{Training dynamics of \ex{}-FLAD, a case study using RTE as target dataset and T0Mix as auxiliary data. Colored lines are a sample of auxiliary datasets with interesting properties, the remaining datasets are shown in grey.}
    \label{fig:ex_rte}
\end{figure*}

\begin{figure*}[t]
    \centering
    \includegraphics[width=0.95\textwidth]{images/UCB1_dynamics_copa.pdf}
    \caption{
    Training dynamics of \ucb{}-FLAD, a case study using COPA as target dataset and T0Mix as auxiliary data. Colored lines are a sample of auxiliary datasets with interesting properties, the remaining datasets are shown in grey.
    We find that although qasc and quartz start with very high gradient alignment, they very quickly fall to below - 0 alignment (middle figure, green and yellow). In the end, we find that the algorithm samples much more from qasc than from quartz (bottom figure).
    Interestingly, we find that although both cnn\_dailymail and multi\_news start off with very negative gradient alignment, they quickly become the most aligned with the target task (middle figure, blue and red).
    We find that the three auxiliary datasets with highest upper confidence index (top figure) and largest sampling percent (bottom figure) are cnn\_dailymail, multi\_news, and trec even though these all considered dissimilar to the target prior to training.
    }
    \label{fig:ucb_copa}
\end{figure*}


\begin{figure*}[t]
    \centering
    \includegraphics[width=0.95\textwidth]{images/EXP3_dynamics_copa.pdf}
    \caption{Training dynamics of \ex{}-FLAD, a case study using COPA as target dataset and T0Mix as auxiliary data. Colored lines are a sample of auxiliary datasets with interesting properties, the remaining datasets are shown in grey.}
    \label{fig:ex_copa}
\end{figure*}



\end{document}


% This document was modified from the file originally made available by
% Pat Langley and Andrea Danyluk for ICML-2K. This version was created
% by Iain Murray in 2018, and modified by Alexandre Bouchard in
% 2019 and 2021 and by Csaba Szepesvari, Gang Niu and Sivan Sabato in 2022. 
% Previous contributors include Dan Roy, Lise Getoor and Tobias
% Scheffer, which was slightly modified from the 2010 version by
% Thorsten Joachims & Johannes Fuernkranz, slightly modified from the
% 2009 version by Kiri Wagstaff and Sam Roweis's 2008 version, which is
% slightly modified from Prasad Tadepalli's 2007 version which is a
% lightly changed version of the previous year's version by Andrew
% Moore, which was in turn edited from those of Kristian Kersting and
% Codrina Lauth. Alex Smola contributed to the algorithmic style files.
