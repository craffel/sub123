\begin{abstract}
% What are we trying to do and why is it relevant?
Few-shot learning involves learning an effective model from only a few labeled datapoints.
The use of a small training set makes it difficult to avoid overfitting but also makes few-shot learning applicable to many important real-world settings.
% where collecting labeled data is difficult.
In this work, we focus on \textbf{F}ew-shot \textbf{L}earning with \textbf{A}uxiliary \textbf{D}ata (FLAD), a training paradigm that assumes access to auxiliary data during few-shot learning in hopes of improving generalization.
% Why is this hard?
Introducing auxiliary data during few-shot learning leads to essential design choices
% , such as determining how to mix auxiliary and target data,
where hand-designed heuristics can lead to sub-optimal performance. 
% How do we solve it? What's our contribution
In this work, we focus on automated sampling strategies for FLAD and relate them to the explore-exploit dilemma that is central in multi-armed bandit settings.
Based on this connection we propose two algorithms -- \ex{}-FLAD and \ucb{}-FLAD -- and compare them with methods that either explore or exploit, finding that the combination of exploration \textit{and} exploitation is crucial.
% How do we verify that our contributions have solved it?
Using our proposed algorithms to train T5 yields a 9\% absolute improvement over the explicitly multi-task pre-trained T0 model across 11 datasets.
\alon{Mention something about analysis here.}
\end{abstract}
