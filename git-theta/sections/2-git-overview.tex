\section{Version Control with Git}

Git is an open-source version control system designed in 2005 to support collaborative development of the Linux kernel. Since then, Git has emerged as one of the most widely used version control systems, particularly in the open-source software community \cite{TODO}.

In the typical Git workflow, teams host a main branch---a sequence of repository snapshots---on a centralized Git remote server (e.g., Github, GitLab, BitBucket, etc.). 
Team members can independently contribute changes to the main branch by downloading a full copy of the repository from the remote (\code{git fetch}), committing changes locally (\code{git commit}), and sending their changes back to the remote repository (\code{git push}).

Commonly, contributors to a Git repository will create a new local branch when making changes (\code{git branch}). Creating a branch allows one contributor to make a series of commits in isolation from any concurrent changes that other contributors make on the main branch. Additionally, branching makes switching between logically-separate features easy (\code{git checkout}). Once development on a branch is done, its changes are merged (\code{git merge}) in to the main branch, adding a new commit to the main branch's history. Additionally, Git provides several resolution methods for merging branches that contain changes to the same files. 

Internally, Git manages repositories by storing files in three logically separate locations. First is the \emph{working tree}, where files in a repository are viewed and edited by users. The second location is the \emph{staging area}, a location managed internally by Git that contains modified files that are ready to be committed. Files are placed in the staging area by Git when users run \code{git add}. Finally, after staged files are committed and a user runs \code{git push}, the committed files are stored on the \emph{Git remote}. \nikhil{Maybe reference a Figure here}

% Git is an open-source version control system that tracks changes to a file via a series of file snapshots. While Git is distributed---each user retains a full copy of the repository history---it is commonly used in a centralized fashion. Services like GitHub offer a place to host a ``remote'' copy of a git repository. Team members can work independently, making commits in their local repository, download other's changes (\code{git pull}), and send their own changes to the remote repository (\code{git push}), where any conflicting changes are resolved. 

% Files live in 3 places in Git. The first place is the working directory---this is where the file can be edited is checked for modifications. The second place is the ``staging area''. Essentially a waiting area that makes it easier to specify which modified files are actually going to be part of the next commit. Files are moved to the staging area with the \code{git add} command. Running \code{git commit} moves files from the staging area into the local git database and creates a commit.

% Git began in 2005 to support the non-linear development workflow used by the Linux kernel developers. Non-linear development is generally handled via branches and merges. Creating a branch allows one to create a series of commits, implementing their changes, starting from a fixed point in history and in isolation from changes others make. Branching makes switching between logically-separate features easy and stops changes that land on the main branch from effecting your development. Once a development on a branch is done, its changes are ``merged'' back in to the main branch, creating a commit that includes changes from both branched. Additionally, Git provides several resolution methods when disparate branches makes incompatible changes to the same file.

